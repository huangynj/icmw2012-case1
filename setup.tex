\documentclass{article}
\usepackage{natbib,hyperref,vmargin,amsmath,pgf}


\title{
  8\textsuperscript{th} International Cloud Modelling Workshop\\
  Case 1: CCN processing by a drizzling stratocumulus
}
\author{
  Organized by Wojciech Grabowski (NCAR/MMM) and Zach Lebo (NCAR/ASP)
}
\date{
  version of: \today
}

\begin{document}

  \maketitle

  \section{Overview}

  The purpose of this case is to investigate accuracy and efficiency of various 
    approaches to simulate processing of CCN by precipitating clouds. 
  The motivation comes from observed dramatic differences between aerosol 
    characteristics between areas of closed and open cells within stratocumulus 
    over subtropical eastern Pacific. 
  The problem is important because open and closed cells have dramatically 
    different cloud characteristics, with almost 100\% cloud cover for the closed 
    cells and significantly lower (below 20\%) cloud cover for the open cells. 
  CCN processing at the boundary between open and closed cells has been argued to 
    play an important role, perhaps even control, the transition from open- 
    to closed-cell circulations.

  \section{Case description}

  The specific modeling case described here is based on observations 
    of drizzling stratocumulus. 
  Two separate modeling approaches are considered, the prescribed-flow 
    (kinematic model) case, and the two-dimensional (2D) dynamic airflow model, 
    both focusing on microphysical processes within drizzling stratocumulus. 
  Each participant is expected to use both frameworks and prepare the results 
    using the guidelines discussed below. 
  However, using only one of the frameworks is also acceptable.

  The setup for both dynamic and kinematic cases is based on VOCALS observation 
    of subtropical drizzling stratocumulus. 
  Idealized profiles of the total water and liquid water potential temperature 
    are used to specify the initial profiles for the model simulations. 
  Initial CCN characteristics are also base on VOCALS observations. 
  In both kinematic and dynamic model simulations, the emphasis is on the 
    representation of CCN processing by the drizzling stratocumulus and on the 
    gradual increase of the drizzle rate as a result of reduction of total 
    CCN number and modification of CCN size distribution.

  The initial liquid water potential temperature, $\theta_l$, and the total water 
    mixing ratio profiles, $q_t$ are prescribed as:
    \begin{equation*}
      \theta_l = \begin{cases}
        289.0 & \mbox{for~} z \le z^0_{inv}\\
        303.0 + (z - z^0_{inv})^{1/3} & \mbox{for~} z > z^0_{inv}
      \end{cases}
    \end{equation*}
    \begin{equation*}
      q_t = \begin{cases}
        7.5 & \mbox{for~} z \le z^0_{inv}\\
        0.5 & \mbox{for~} z > z^0_{inv}
      \end{cases}
    \end{equation*}
    where $z^0_{inv} = 1500 m$ is the initial inversion height. 
  Note that $z^0_{inv}$ is also the depth of the kinematic model computational domain 
    and only profiles below that height (i.e., constant $\theta_l$ and $q_t$) are used 
    as initial profiles in the kinematic model. 
  When decomposing the profiles into profiles of the potential temperature, water vapor 
    and cloud water mixing ratio, the hydrostatic pressure profile has to be used assuming 
    the surface pressure of 1015 hPa. 
  The initial profiles represent the well-mixed stratocumulus-topped boundary layer with 
    a cloud extending from the cloud base at approximately 950 m to the boundary-layer 
    inversion at 1500 m.

  For simplicity we assume that the aerosols are amonium sulfates with the initial dry 
    aerosol spectrum given by two-mode lognormal aerosol size distribution following 
    observations reported in \citet[][Figs.~15 and 16 therein]{Wood_et_al_2011} and 
    \citet[][Table~4 therein]{Allen_et_al_2011}, but with low concentrations to initiate 
    quickly aerosol processing. 
  The parameters for the initial distributions for the the two modes are the total 
    concentrations of 60 and 40 cm\textsuperscript{-3}, mean radii of 0.04 and 0.15~$\mu$m, 
    and geometric standard deviations of 1.4 and 1.6 for the first and second mode, respectively.

  \section{Kinematic Model simulations}

  The kinematic model computational domain is 1500~x~1500 m\textsuperscript{2}, the gridlength 
    is 20~m in both directions, and the grid is 75~x~75. 
  The prescribed velocity field represents a single eddy spanning the entire extent of the 
    computational domain and featuring the maximum vertical velocity of about 1~m~s\textsuperscript{-1}, 
    see Fig.~\ref{fig1}. 
  To maintain steady mean temperature and moisture profiles (i.e., to compensate for gradual water loss 
    due to precipitation and warming of the boundary layer due to latent heating), 
    mean temperature and moisture profiles are relaxed to the initial profiles with a height-dependent 
    relaxation time scale $\tau = \tau_{0}{\rm exp}(z/z_{rlx})$, with $\tau_{0} = 300 s$ and $z_{rlx} = 200 m$
    (i.e., temperature and moisture equations have an additional source in the form $-(-<\psi> - \psi_{init})/\tau_{rlx}$
    where $\psi_{init}$ is the initial profile and $<\psi>$ is the horizontal mean of $\psi$ at a given height. 
  Note that such a formulation does not dump small-scale perturbations of $\psi$, but simply shifts the 
    horizontal mean toward $\psi_i$. 
  Such a term mimics the effects of surface fluxes that maintain the balance in the natural case.

  The initial simulations should only consider activation/coalescence scavenging, 
    that is, the change of CCN spectrum due to drizzle should only include processes 
    associated with CCN activation, droplet collisional growth, drizzle fallout and either 
    below-cloud evaporation or loss when reaching the surface. 
  Additional processes, such as, for instance, scavenging of unactivated (interstitial) 
    CCN within the cloud (through the interaction with cloud droplets) or scavenging by 
    falling drizzle below the cloud, may be added to compare their impact with the 
    activation/coalescence scavenging. 
  Some discussion of these processes is given in \citet{Flossmann_and_Wobrock_2010}, 
    although not focusing specifically on stratocumulus.

  We anticipate that a variety of schemes will be put to the test of the kinematic model. 
  Arguably the benchmark will be provided by a two-dimensional bin scheme 
    (droplet radius times aerosol mass). 
  Bin model with simplified representation of CCN will be used by some. 
  A double-moment microphysics with possibly double-moment CCN representation will serve as a 
    computationally-efficient approach, and exploring how accurate such an approach is to represent 
    aerosol processing is one of particular goals of this case.

  The FORTRAN~77 source code linked here\footnote{\url{https://raw.github.com/slayoo/icmw2012-case1/master/kinematic_wrain.vocals.f}} 
    can be used for the kinematic model and a new modular microphysics scheme can be readily replaced 
    into this code by individual developers of similar routines. 
  It has a bulk representation of the microphysics without CCN processing, 
    and has to be adopted to a specific microphysics scheme. 
  The code has many comments and it should be easy to modify it. 
  This code also contains calls to NCAR-Graphics for plotting purposes. 
  Either link to proper NCARG libraries or comment out the graphical portions of code.
  Please email W.~Grabowski (grabow@ucar.edu) if you encounter any problems. 

  Example of results with a simple 1-moment bulk scheme without considering CCN processing 
    obtained with the code available for download is shown in Fig.~\ref{fig2}. 
  The figure shows the quasi-steady solution from the kinematic model that is reached 
    after a few hours (solution after one hour differs from the one showed in small details). 
  As the figure shows, drizzle does not reach the surface in this case, but it should lead 
    to changes in the CCN spectrum once such effects are included.

  \clearpage

  \section{Dynamic Model simulations}

  The dynamic model simulations should include key processes affecting 
    the dynamics of the strato-cumulus-topped boundary layer (STBL), 
    such as the cloud-top radiative cooling and surface water and energy 
    fluxes that drive STBL circulations. 
  It is left to the user to select the simulations framework (i.e., 2D versus 3D, 
    specific domain size and gridlength). 
  Below we present a setup that we used in a 2D simulation applying bulk 
    microphysics without CCN processing. 
  It is suggested that such a setup is used first to provide comparison with 
    other participants. 
  Subsequent tests should then explore the role of dimensionality, 
    domain size, gridlength size, etc.

  For the 2D dynamic model simulation, the computational domain is taken as 
    5~km in the horizontal (X) and 3~km in the vertical (Z).
   Model gridlength is taken to be (50,20) m in (X,Z). 
  The model is driven by the cloud-top radiative cooling and by the surface fluxes. 
  The radiative cooling is treated in a simplified way based on an approach developed 
    by \citet[][eqs. 3 and 4 in particular]{Stevens_et_al_2005}.  
  However, the free-tropospheric cooling is excluded and the local temperature tendency
    is derived from the divergence of the upward and downward fluxes given by
    \begin{equation*}
      F_{rad}(x,z,t) = F_0\,{\rm exp}(-Q(z, \infty) + F_1\,{\rm exp}(-Q(0,z)))
    \end{equation*}
    \begin{equation*}
      Q(a,b) = \kappa \int\limits_a^b \rho (q_c + q_r) dz
    \end{equation*}
    where, with $F_0 = 113 \,W m^{-2}$, $F_1 = 22 \,W m^{-2}$, $\kappa = 85 \,m^2 kg^{-1}$ 
    (i.e., the latter two same as in \citet{Stevens_et_al_2005}, and $\rho$ being 
    the air density. 
  Note that $F_0$ was increased from the value used in \citet{Stevens_et_al_2005} 
    (70 W m\textsuperscript{-2}) per Chris Bretherton's suggestion to match 
    observations from VOCALS flights.

  We purposely exclude large-scale subsidence that controls the depth of the boundary 
    layer to avoid numerical issues with the subsidence implementation into the model. 
  The increase of the boundary layer height can then be used as one of the measures 
    of the simulated entrainment rate. 
  This is consistent with excluding the free-tropospheric cooling included in 
    \citet{Stevens_et_al_2005} and omitted here. 
  Surface latent and sensible heat fluxes are prescribed as 115 and 15 W m\textsuperscript{-2}, 
    respectively. 
  To ensure that the mean temperature and moisture profiles stay close to the initial profiles, 
    a lower-boundary-layer relaxation similar to that used in the kinematic model is also applied 
    in the dynamic model, but with a longer relaxation time scale, $\tau_{rlx} = 1800 s$. 
  Finally, to initiate boundary-layer circulations from motionless horizontally-uniform initial 
    state, random temperature and moisture perturbations within the boundary layer 
    (i.e., beneath 1500 m height) should be added. 
  The amplitudes of the random perturbations should be 0.1~K and 0.2~g~kg\textsuperscript{-1}
    for the potential temperature and water vapor mixing ratio.

  The dynamic model is supposed to be run for 12 hours. 
  It is suggested that the model is run for the first hour without drizzle formation 
    to allow spinup of boundary-layer eddies.

  Otherwise, significant drizzle develops at the onset in almost motionless atmosphere 
    and leads to rapid thinning of the cloud layer. 
  Note that this case features initially relatively deep boundary layer and deep cloud layer. 
  Such conditions lead to the decoupling of the boundary layer \citep[cf.][]{Jones_et_al_2011}, 
    an aspect that might play significant role in the dynamic model simulations, in contrast 
    to the kinematic model where the flow is prescribed. 
  Another aspect is that the interactions between dynamics and microphysics lead to solutions 
    with significant temporal fluctuations of mean properties, as illustrated in Fig.~\ref{fig3}. 
  This is in contrast to the kinematic model results and may have some impact on the comparison 
    between results using different microphysical schemes. 
  Arguably, such fluctuations should be less pronounced in 3D models.

  The emphasis of the analysis should be on the evolution of mean properties characterizing 
    cloud and precipitation processes, such as the mean cloud depth and cloud water path, 
    the mean cloud droplet concentration and surface precipitation rate, as well as the 
    evolution of the CCN characteristics (mean concentration, mean radius and mass, etc). 
  Note that possibly significant difference between kinematic and dynamic model simulations 
    in the cloud-top entrainment, present only in the dynamic model simulations. 
  This aspects can be one of the foci of the intercomparison between results from the two models.

  \begin{figure}
    \pgfimage[width=\textwidth]{fig1}
    \caption{\label{fig1}
    }
  \end{figure}

  \begin{figure}
    \pgfimage[width=\textwidth]{fig2}
    \caption{\label{fig2}
    }
  \end{figure}
  
  \begin{figure}
    \pgfimage[width=\textwidth]{fig3}
    \caption{\label{fig3}
    }
  \end{figure}
  
  \clearpage

  \bibliographystyle{copernicus}
  \bibliography{setup}

\end{document}
